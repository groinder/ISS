\documentclass[a4paper,10pt]{article}
\usepackage[utf8]{inputenc}
\usepackage{polski}
\usepackage{graphicx}
\usepackage{listings}
\usepackage[usenames,dvipsnames]{color}
\addtolength{\hoffset}{-2cm}
\addtolength{\voffset}{-3cm}
\addtolength{\textwidth}{4cm}
\addtolength{\textheight}{5cm}
\usepackage{setspace}
\usepackage{indentfirst}
\usepackage{graphicx}
\lstset{
    language=Matlab,
    basicstyle=\scriptsize,
    aboveskip={1.5\baselineskip},
    columns=fixed,
    showstringspaces=false,
    extendedchars=true,
    breaklines=true,
    tabsize=4,
    prebreak = \raisebox{0ex}[0ex][0ex]{\ensuremath{\hookleftarrow}},
    frame=single,
    showtabs=false,
    showspaces=false,
    showstringspaces=false,
    identifierstyle=\ttfamily,
    keywordstyle=\color[rgb]{0,0,1},
    commentstyle=\color[rgb]{0.133,0.545,0.133},
    stringstyle=\color[rgb]{0.627,0.126,0.941},
    numbers=left,
    numberstyle=\tiny,
    stepnumber=1,
    numbersep=5pt,
    captionpos=b,
    escapeinside={\%*}{*)}
}

\def\figurename{Rys.}
\def\lstlistingname{Fun.}

\title{Informatyczne Systemy Sterowania \\ \large Ćwiczenie 2: Systemy regulacji - Regulator PID}

\author{Krzysztof Przybylski 239266}

\begin{document}
\maketitle

\section{Wstęp}\label{sec:wstęp}
\subsection{Cel ćwiczenia}
Celem tego ćwiczenia jest poznanie podstawowej struktury systemu sterowania z typową formą algorytmu regulacji PID, oraz zapoznanie się z częścią pakietu M\small ATLAB \normalsize - Simulinkiem, oraz jego możliwości w zakresie modelowania i analizy systemów regulacji.
%Więcej?

\subsection{Plan badań} 
\begin{enumerate}
	\item Symulacja systemu regulacji
	\begin{enumerate}
		\item Regulator P. Zbadać wpływ wartości wzmocnienia regulatora na działanie systemu. 
	    \item Regulator   PI.   Zbadać  wpływ   wartości   parametru   całkowania   na   działanie   systemu 
       (dla ustalonej wartości parametru $k$). 
    	\item Regulator PID. Zbadać wpływ wartości parametru różniczkowania na działanie 
       systemu (dla ustalonych wartości parametrów $k$ i $T_i$). 
	\end{enumerate}
	\item Dobór optymalnych parametrów regulatora. Sporządzenie wykresów:
	\begin{enumerate}
		\item dla regulatora P. 
    	\item dla regulatora PI przy trzech różnych ustalonych wartościach parametru $k$.  
    	\item dla regulatora PID przy trzech różnych wartościach parametru oraz ustalonej wartości parametru $k$.
	\end{enumerate}
	\item Dobór parametrów regulatora PID według zasad Zieglera–Nicholsa
	\begin{enumerate}
		\item Doświadczalnie znaleźć współczynnik wzmocnienia, dla którego układ traci stabilność.
		\item Ustalić okres oscylacji i wzmocnienie krytyczne.
		\item Według odpowiednich rekomendacji określić wartości parametrów regulatora PID.
	\end{enumerate}
\end{enumerate}
Zadania należy wykonać dla obiektu o podanej transmitancji operatorowej:
\begin{equation} \label{eqn:transOS}
	G(s) = \frac{b}{s^3 + a_2 s^2 + a_1 s + a_0}
\end{equation}
%Copy/Paste z treści zadania...

\newpage
\section{Realizacja planu i wyniki}

%---------------------------------------------------------------------------------------------------------------------
%
%ZADANIE 1
%
%---------------------------------------------------------------------------------------------------------------------
\subsection{Symulacja systemu regulacji.}\label{sec:zad1}
System regulacji będziemy symulować przy użyciu programu Simulink będącego częścią pakietu M\small ATLAB. \\
\normalsize Schemat systemu służącego nam do symulacji przedstawiony został na poniższym rysunku.

\begin{figure}[!h]
    \centering
	\includegraphics[width=120mm]{schemat.png}
	\caption{Schemat symulacji obiektu z regulatorem PID}
    \label{fig:Rysunek}
\end{figure}

Wartości błędu regulacji $\varepsilon(t)$ mogą być odczytywane zarówno na wykresie $Scope$ jak i z poziomu M\small ATLABA \normalsize dzięki obiektowi $simout$.

Regulator PID jest opisany następującą transmitancją:
\begin{equation} \label{eqn:transPID}
	G(s) = k + {T_{i} \over s} + T_{d}s ,
\end{equation}
gdzie $k$ to współczynnik wzmocnienia, $T_{i}$ to parametr członu całkującego, a $T_{d}$ to parametr członu różniczkującego.

Podczas ćwiczenia przyjęliśmy następujące wartości parametrów transmitancji obiektu sterowania (\ref{eqn:transOS}): 
\begin{equation} \label{eqn:paramTransOS}
	b = 1,
	a_{2} = 5, 
	a_{1} = 1, 
	a_{0} = 3
\end{equation}
oraz zadanej wartości wyjścia $y^{*} = 1$.

\subsubsection{Regulator P.}\label{sec:regP}
Aby zasymulować regulator P musieliśmy ustawić wartości parametru całkującego i różniczkującego na $T_{i}=0$, $T_{d}=0$. \\
Następnie wykorzystując poniższą funkcję przetestowaliśmy zachowanie regulatora P dla różnych wartości parametru $k$. \\
\begin{lstlisting}[caption=Funkcja testująca regulator P.]
function testP(start, step, stop)
load_system('pidModel.mdl');
hold on;
k = start;
color = char('g', 'm', 'r', 'k', 'b', 'y');
set_param('pidModel/PID Controller', 'I', num2str(0));
set_param('pidModel/PID Controller', 'D', num2str(0));
i = 0;
legend('on');
while (k <= stop)
	set_param('pidModel/PID Controller', 'P', num2str(k));
	sim('pidModel.mdl');
	figure(1);
	plot(simout.time, simout.signals.values, 'Color', color(mod(i,6)+1), 'DisplayName', strcat('k=',num2str(k)));
	i = i + 1;
	k = k + step;
end
end
\end{lstlisting}
W ten sposób otrzymalismy następujące wykresy: \\
\begin{figure}[!h]
    \centering
	\includegraphics[width=130mm]{P-ku.png}
	\caption{Wykres przebiegu funkcji $\varepsilon(t)$ regulatora P dla wartości ujemnych k.}
    \label{fig:regulatorPeu}
\end{figure}
\begin{figure}[!h]
    \centering
	\includegraphics[width=130mm]{P-kd.png}
	\caption{Wykres przebiegu funkcji $\varepsilon(t)$ regulatora P dla wartości dodatnich k.}
    \label{fig:regulatorPed}
\end{figure}
\\ Z wykresu \ref{fig:regulatorPeu}, oraz \ref{fig:regulatorPed} możemy zauważyć, że:
\begin{itemize}
	\item Dla ujemnych wartości parametru $k$ drgania po pewnym czasie ustają, a wykres $\varepsilon(t)$ stabilizuje się na pewnym poziomie większym od żądanej wartości $y*$, a poziom ten jest tym większy im mniejsza jest wartość parametru $k$.
	\item Dla wartości dodatnich wykres $\varepsilon(t)$ przyjmuje postać drgań oscylujących wokół wartości ~0.6. Dla wartości $k$ mniejszych od 2 amplituda tych drgań maleje, natomiast rośnie dla wartości większych od 2. Dla $k=2$ amplituda tych drgań jest stała.
	\item Regulator P nie jest efektywny, ponieważ wykres funkcji $\varepsilon(t)$ oscyluje poniżej wartości $y*$, ale nie wokół 0.
\end{itemize}
\subsubsection{Regulator PI.}\label{sec:regPI}
Aby zasymulować regulator PI musieliśmy ustawić wartości parametru proporcjonalnego i różniczkującego na $k=2$, $T_{d}=0$. \\
Następnie wykorzystując poniższą funkcję przetestowaliśmy zachowanie regulatora PI dla różnych wartości parametru $T_{i}=0$. \\
\begin{lstlisting}[caption=Funkcja testująca regulator PI.]
function testPI(start, step, stop)
load_system('pidModel.mdl');
hold on;
ti = start;
i = 0;
color = char('g', 'm', 'r', 'k', 'b', 'y');
set_param('pidModel/PID Controller', 'P', num2str(2));
set_param('pidModel/PID Controller', 'D', num2str(0));
legend('on');
while (ti <= stop)
	set_param('pidModel/PID Controller', 'I', num2str(ti));
	sim('pidModel.mdl');
	figure(1);
	plot(simout.time, simout.signals.values, 'Color', color(mod(i,6)+1), 'DisplayName', strcat('k=',num2str(ti)));
	i = i + 1;
	ti = ti + step;
end
end
\end{lstlisting}
W ten sposób otrzymalismy następujące wykresy: \\
\begin{figure}[!h]
    \centering
	\includegraphics[width=130mm]{PI-ku.png}
	\caption{Wykres przebiegu funkcji $\varepsilon(t)$ regulatora PI dla wartości ujemnych $T_{i}$.}
    \label{fig:regulatorPIeu}
\end{figure}
\begin{figure}[!h]
    \centering
	\includegraphics[width=130mm]{PI-kd.png}
	\caption{Wykres przebiegu funkcji $\varepsilon(t)$ regulatora PI dla wartości dodatnich $T_{i}$.}
    \label{fig:regulatorPIed}
\end{figure}
\newpage Z wykresu \ref{fig:regulatorPIeu}, oraz \ref{fig:regulatorPIed} możemy zauważyć, że:
\begin{itemize}
	\item Dla ujemnych wartości parametru $T_{i}$ wartości funkcji $\varepsilon(t)$ szybko rosną. Tempo w jakim wzrasta jest tym większe im mniejszą wartość ma parametr $T_{i}$.
	\item Dla wartości dodatnich wykres $\varepsilon(t)$ przyjmuje postać drgań, które z początku oscylują tak jak przy regulatorze P, lecz z czasem oscylują wokół wartości coraz bliższej 0. Tempo w jakim zbliża się do 0 jest tym większe im większą wartość ma parametr $T_{i}$. Amplituda drgań z czasem wzrasta.
	\item Regulator PI jest bardziej efektywny niż regulator P, ponieważ wykres funkcji $\varepsilon(t)$ oscyluje bliżej 0.
\end{itemize}
\subsubsection{Regulator PID.}\label{sec:regPID}
Aby zasymulować regulator PID musieliśmy ustawić wartości parametru proporcjonalnego i całkujacego na $k=2$, $T_{i}=1$. \\
Następnie wykorzystując poniższą funkcję przetestowaliśmy zachowanie regulatora P dla różnych wartości parametru $T_{d}$. \\
\begin{lstlisting}[caption=Funkcja testująca regulator PID.]
function testPID(start, step, stop)
load_system('pidModel.mdl');
hold on;
td = start;
i = 0;
color = char('g', 'm', 'r', 'k', 'b', 'y');
set_param('pidModel/PID Controller', 'P', num2str(2));
set_param('pidModel/PID Controller', 'I', num2str(1));
legend('on');
while (td <= stop)
	set_param('pidModel/PID Controller', 'D', num2str(td));
	sim('pidModel.mdl');
	figure(1);
	plot(simout.time, simout.signals.values, 'Color', color(mod(i,6)+1), 'DisplayName', strcat('k=',num2str(td)));
	i = i + 1;
	td = td + step;
end
end
\end{lstlisting}
W ten sposób otrzymalismy następujące wykresy: \\
\begin{figure}[!h]
    \centering
	\includegraphics[width=130mm]{PID-ku.png}
	\caption{Wykres przebiegu funkcji $\varepsilon(t)$ regulatora PID dla wartości ujemnych $T_{d}$.}
    \label{fig:regulatorPIDeu}
\end{figure}
\begin{figure}[!h]
    \centering
	\includegraphics[width=130mm]{PID-kd.png}
	\caption{Wykres przebiegu funkcji $\varepsilon(t)$ regulatora PID dla wartości dodatnich $T_{d}$.}
    \label{fig:regulatorPIDed}
\end{figure}
\newpage Z wykresu \ref{fig:regulatorPIDeu}, oraz \ref{fig:regulatorPIDed} możemy zauważyć, że:
\begin{itemize}
	\item Dla ujemnych wartości parametru $k$ wykres $\varepsilon(t)$ przyjmuje postać drgań, których amplituda z czasem rośnie. Tempo tego wzrostu jest tym większe im mniejszą wartość ma parametr $T_{d}$.
	\item Dla wartości dodatnich wykres $\varepsilon(t)$ przyjmuje postać drgań, które oscylują tak jak przy regulatorze PI, lecz z czasem amplituda maleje. Tempo w jakim ona maleje tym większe im większą wartość ma parametr $T_{d}$.
	\item Regulator PID jest efektywny, ponieważ wykres funkcji $\varepsilon(t)$ oscyluje bliżej 0 niż regulator P, a amplituda drgań, która w regulatorze PI się zwiększała tutaj zanika.
\end{itemize}

%---------------------------------------------------------------------------------------------------------------------
%
%ZADANIE 2
%
%---------------------------------------------------------------------------------------------------------------------
\subsection{Dobór optymalnych parametrów regulatora.}\label{sec:zad2}
Przy doborze optymalnych parametrów przyjęliśmy następujące kryterium:
\begin{equation} \label{eqn:transOS}
	Q = {1 \over n}\sum_{i=1}^{n} \varepsilon(t)
\end{equation}
Gdzie $n$ jest ilością pomiarów wykonanych przy symulacji.

\subsubsection{Regulator P.}\label{sec:optP}
Aby dobrać optymalną wartość parametru $k$ ustawiliśmy parametry podobnie jak w zadaniu \ref{sec:regP}, oraz wykorzystując poniższą funkcję przetestowaliśmy zachowanie regulatora P dla wielu wartości parametru $k$. \\
\begin{lstlisting}[caption=Funkcja testująca regulator P.]
function optimumP(start, step, stop)
load_system('pidModel.mdl');
hold on;
k = start;
i = 0;
set_param('pidModel/PID Controller', 'I', num2str(0));
set_param('pidModel/PID Controller', 'D', num2str(0));
while (k <= stop)
	set_param('pidModel/PID Controller', 'P', num2str(k));
	sim('pidModel.mdl');
	wy = simout.signals.values;
	i = i + 1;
	q(i) = sum(wy.^2)/length(wy);
	ka(i) = k;
	k = k + step;
end
figure(1);
plot(ka, q);
end
\end{lstlisting}

\newpage
W ten sposób otrzymalismy następujące wykresy: \\
\begin{figure}[!h]
    \centering
	\includegraphics[width=130mm]{P-opt.png}
	\caption{Wykres przebiegu funkcji $Q(k)$ regulatora P.}
    \label{fig:regulatorPopt1}
\end{figure}
\begin{figure}[!h]
    \centering
	\includegraphics[width=130mm]{P-opt-zoom.png}
	\caption{Wykres przebiegu funkcji $Q(k)$ regulatora P (powiększenie).}
    \label{fig:regulatorPopt2}
\end{figure}
\\ Wartością optymalną parametru $k$ jest minimum funkcji $Q(k)$, więc z wykresu \ref{fig:regulatorPopt1}, oraz \ref{fig:regulatorPopt2} możemy odczytać, że $k = 2$.

\subsubsection{Regulator PI.}\label{sec:optPI}
Aby dobrać optymalną wartość parametru $T_{i}$ ustawiliśmy parametry podobnie jak w zadaniu \ref{sec:regPI}, oraz wykorzystując poniższą funkcję przetestowaliśmy zachowanie regulatora P dla wielu wartości parametru $T_{i}$ i trzech różnych wartości $k$. \\
\begin{lstlisting}[caption=Funkcja testująca regulator PI.]
function optimumPI(start, step, stop)
load_system('pidModel.mdl');
hold on;
ti = start;
i = 0;
set_param('pidModel/PID Controller', 'D', num2str(0));
set_param('pidModel/PID Controller', 'P', num2str(4));
while (ti <= stop)
	set_param('pidModel/PID Controller', 'I', num2str(ti));
	sim('pidModel.mdl');
	wy = simout.signals.values;
	i = i + 1;
	q(i) = sum(wy.^2)/length(wy);
	tei(i) = ti;
	ti = ti + step;
end
figure(1);
plot(tei, q);
end
\end{lstlisting}
W ten sposób otrzymalismy następujące wykresy: \\
\begin{figure}[!h]
    \centering
	\includegraphics[width=130mm]{PI-opt-k1.png}
	\caption{Wykres przebiegu funkcji $Q(T_{i})$ regulatora PI dla $k=1$.}
    \label{fig:regulatorPIk1opt1}
\end{figure}

\newpage

\begin{figure}[!h]
    \centering
	\includegraphics[width=130mm]{PI-opt-k1-zoom.png}
	\caption{Wykres przebiegu funkcji $Q(T_{i})$ regulatora PI dla $k=1$ (powiększenie).}
    \label{fig:regulatorPIk1opt2}
\end{figure}
\begin{figure}[!h]
    \centering
	\includegraphics[width=130mm]{PI-opt-k2.png}
	\caption{Wykres przebiegu funkcji $Q(T_{i})$ regulatora PI dla $k=2$.}
    \label{fig:regulatorPIk2opt1}
\end{figure}

\newpage

\begin{figure}[!h]
    \centering
	\includegraphics[width=130mm]{PI-opt-k2-zoom.png}
	\caption{Wykres przebiegu funkcji $Q(T_{i})$ regulatora PI dla $k=2$ (powiększenie).}
    \label{fig:regulatorPIk2opt2}
\end{figure}
\begin{figure}[!h]
    \centering
	\includegraphics[width=130mm]{PI-opt-k4.png}
	\caption{Wykres przebiegu funkcji $Q(T_{i})$ regulatora PI dla $k=4$.}
    \label{fig:regulatorPIk4opt1}
\end{figure}
\begin{figure}[!h]
    \centering
	\includegraphics[width=130mm]{PI-opt-k4-zoom.png}
	\caption{Wykres przebiegu funkcji $Q(T_{i})$ regulatora PI dla $k=4$ (powiększenie).}
    \label{fig:regulatorPIk4opt2}
\end{figure}
\newpage Wartością optymalną parametru $T_{i}$ jest minimum funkcji $Q(T_{i})$, więc z wykresów \ref{fig:regulatorPIk1opt1}, \ref{fig:regulatorPIk1opt2}, \ref{fig:regulatorPIk2opt1}, \ref{fig:regulatorPIk2opt2}, \ref{fig:regulatorPIk4opt1}, oraz \ref{fig:regulatorPIk4opt2} możemy odczytać, że dla $k=1, T_{i} \approx 0.14$, dla $k=2, T_{i} \approx 0.06$, natomiast dla $k=4, T_{i} \approx -0.15$.

\subsubsection{Regulator PID.}\label{sec:optPID}
Aby dobrać optymalną wartość parametru $T_{d}$ ustawiliśmy parametry podobnie jak w zadaniu \ref{sec:regPID}, oraz wykorzystując poniższą funkcję przetestowaliśmy zachowanie regulatora P dla wielu wartości parametru $T_{d}$ i trzech różnych wartości $T_{i}$. \\
\begin{lstlisting}[caption=Funkcja testująca regulator PID.]
function optimumPID(start, step, stop)
load_system('pidModel.mdl');
hold on;
td = start;
i = 0;
set_param('pidModel/PID Controller', 'P', num2str(2));
set_param('pidModel/PID Controller', 'I', num2str(0.06));
while (td <= stop)
	set_param('pidModel/PID Controller', 'D', num2str(td));
	sim('pidModel.mdl');
	wy = simout.signals.values;
	i = i + 1;
	q(i) = sum(wy.^2)/length(wy);
	tede(i) = td;
	td = td + step;
end
figure(1);
plot(tede, q);
end
\end{lstlisting}

\newpage
W ten sposób otrzymalismy następujące wykresy: \\
\begin{figure}[!h]
    \centering
	\includegraphics[width=130mm]{PID-opt-ti-02.png}
	\caption{Wykres przebiegu funkcji $Q(T_{d})$ regulatora PID dla $T_{i}=0.02$.}
    \label{fig:regulatorPIDti15opt1}
\end{figure}
\begin{figure}[!h]
    \centering
	\includegraphics[width=130mm]{PID-opt-ti-02-zoom.png}
	\caption{Wykres przebiegu funkcji$Q(T_{d})$ regulatora PID dla $T_{i}=0.02$ (powiększenie).}
    \label{fig:regulatorPIDti15opt2}
\end{figure}
\newpage
\begin{figure}[!h]
    \centering
	\includegraphics[width=130mm]{PID-opt-ti-06.png}
	\caption{Wykres przebiegu funkcji $Q(T_{d})$ regulatora PID dla $T_{i}=0.06$.}
    \label{fig:regulatorPIDti35opt1}
\end{figure}
\begin{figure}[!h]
    \centering
	\includegraphics[width=130mm]{PID-opt-ti-06-zoom.png}
	\caption{Wykres przebiegu funkcji $Q(T_{d})$ regulatora PID dla $T_{i}=0.06$ (powiększenie).}
    \label{fig:regulatorPIDti35opt2}
\end{figure}
\newpage
\begin{figure}[!h]
    \centering
	\includegraphics[width=130mm]{PID-opt-ti-1.png}
	\caption{Wykres przebiegu funkcji $Q(T_{d})$ regulatora PID dla $T_{i}=0.1$.}
    \label{fig:regulatorPIDti100opt1}
\end{figure}
\begin{figure}[!h]
    \centering
	\includegraphics[width=130mm]{PID-opt-ti-1-zoom.png}
	\caption{Wykres przebiegu funkcji $Q(T_{d})$ regulatora PID dla $T_{i}=0.1$ (powiększenie).}
    \label{fig:regulatorPIDti100opt2}
\end{figure}
\newpage Wartością optymalną parametru $T_{d}$ jest minimum funkcji $Q(T_{d})$, więc z wykresu \ref{fig:regulatorPIDti15opt1}, \ref{fig:regulatorPIDti15opt2}, \ref{fig:regulatorPIDti35opt1}, \ref{fig:regulatorPIDti35opt2}, \ref{fig:regulatorPIDti100opt1}, oraz \ref{fig:regulatorPIDti100opt2} możemy odczytać, że dla $T_{i}=0.02$, $T_{d} \approx  466.1$, dla $T_{i}=0.06$, $T_{d} \approx 459.8$, oraz dla $T_{i}=0.1, T_{d} \approx 447,7$.

\subsection{Zastosowanie zasad Zieglera-Nicholsa.}\label{sec:zad3}
Doświadczalnie znalezienie współczynnika wzmocnienia dla którego układ traci stabilność.
Ustalenie Okresu oscylacji oraz wzmocnienia krytycznego.
Określić wartości parametrów regulatora PID.\\

\noindent a. Doświadczalnie znaleźć współczynnik wzmocnienia, dla którego układ traci stabilność. \\

\noindent $p = 2$ \\

\noindent b. Ustalić okres oscylacji i wzmocnienie krytyczne. \\

\noindent $k_{p} = 2$, 
$k_{u} = 6$ \\

\noindent c. Według odpowiednich rekomendacji określić wartości parametrów regulatora PID.\\

\noindent $P = k_{p} * 0.6 = 1,2$ \\
$I = k_{u}  * 0.5 = 3$ \\
$D = k_{u} * 0.125 = 0.75$

\section{Wnioski}
Podczas wykonywania ćwiczenia zauważyłem pewne właściwości parametrów oraz w jaki sposób ich zmiana wpływa na regulator PID. Sterowanie proporcjonalne (k) ma wpływ na czas narastania (zmniejsza ten czas) oraz błąd sterowania (zmniejsza go), ale nie daje dobrego efektu. 
Natomiast sterowanie całkujące (Ti) ma wpływ na eliminowanie błędu sterowania w stanie ustalonym, jednak pogarsza odpowiedź w czasie narastania i przed czasem ustalonym, co niewątliwie jest wadą.
Sterowanie różniczujące (Td) zwiększa stabilność układu poprzez zmniejszenie przeregulowania i poprawe odpowiedzi między stanami. Odpowiedni odbór tych paramatrów jest bardzo ważny dla układu regulacji, gdyż nieodpowiednie ich dobranie może powodować nieprawidłowe działanie lub w szczególych przypadkach destabilizację układu.

\end{document}